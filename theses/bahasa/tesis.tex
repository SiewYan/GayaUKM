%% Contoh tesis GayaUKM dalam Bahasa Melayu
\documentclass[bahasam,nohyphen]{../../packages/GayaUKM}
\usepackage{../../packages/common}
% your thesis title (English)
\newcommand{\mytitle}{The Amazing Thesis that Would Blow \\ Your Frogging Mind}
% Your thesis title (bahasa)
\newcommand{\mytitlems}{Tesis yang Mengagumkan sehingga Menletupkan \\ kepada Otak Kau}
% your name
\newcommand{\myauthor}{Chin Chong Chang}
% your matric ID
\newcommand{\myauthorid}{P88888}
% faculty (English)
\newcommand{\myfaculty}{Faculty of Science and Technology}
% faculty (bahasa)
\newcommand{\myfacultyms}{Fakulti Sains dan Teknology}
% submission data
\newcommand{\mysubmissiondate}{\today}
% year
\newcommand{\mysubmissionyear}{2022}
% degree (English)
\newcommand{\mydegreetype}{Doctor of Philosophy}
% degree (bahasa)
\newcommand{\mydegreetypems}{Doktor Falsafah}
% campus
\campus{Bangi}


\title{\mytitlems}
\author{\myauthor}
\authorid{\myauthorid}
\faculty{\myfacultyms}
\submissiondate{\mysubmissiondate}
\submissionyear{\mysubmissionyear}
\degreetype{\mydegreetype}
\campus{Bangi} %% Assuming Malaysian cities are spelt the same way in English and Malay   

%% If you find the boxes around hyperlinks distracting
\hypersetup{colorlinks,allcolors=black}

\begin{document}

%% Cover page
\makecoverpage

%% Re-specify your title with different manual line breaks for the
%% title page, if necessary
\title{\mytitlems}
\maketitlepage

\frontmatter
\declaration

% penghargaan dari penghargaan.tex
\input{penghargaan}

% abstrak dlm Bahasa Melayu dari abstrak-ms.tex
\input{../../abstracts/abstrak-ms}

% abstrak dlm Bahasa Inggeris dari abstract-en.tex
\input{../../abstracts/abstract-en}


\tableofcontents\clearpage
\listoffigures\clearpage
\listoftables\clearpage

% Senari simbol dll boleh disediakan seperti
% dalam senaraisimbol.tex
\input{../../supplements/symbols}


\mainmatter
% Setiap satu bab dari fail berasingan
\input{bab-1}
\input{bab-2}
\input{bab-3}
\input{bab-4}

% rujukan tersenarai dlm refs.bib
\bibliography{../references}

\appendix
% Setiap satu bab apendiks dari fail berasingan
\input{../../supplements/app-details}
\input{../../supplements/app-code}

\end{document}

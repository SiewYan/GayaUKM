%% Example GayaUKM thesis in English
\documentclass[english,nohyphen]{../../packages/GayaUKM}
\usepackage{../../packages/common}
% your thesis title (English)
\newcommand{\mytitle}{The Amazing Thesis that Would Blow \\ Your Frogging Mind}
% Your thesis title (bahasa)
\newcommand{\mytitlems}{Tesis yang Mengagumkan sehingga Menletupkan \\ kepada Otak Kau}
% your name
\newcommand{\myauthor}{Chin Chong Chang}
% your matric ID
\newcommand{\myauthorid}{P88888}
% faculty (English)
\newcommand{\myfaculty}{Faculty of Science and Technology}
% faculty (bahasa)
\newcommand{\myfacultyms}{Fakulti Sains dan Teknology}
% submission data
\newcommand{\mysubmissiondate}{\today}
% year
\newcommand{\mysubmissionyear}{2022}
% degree (English)
\newcommand{\mydegreetype}{Doctor of Philosophy}
% degree (bahasa)
\newcommand{\mydegreetypems}{Doktor Falsafah}
% campus
\campus{Bangi}


%% Your metadata. Note that an English thesis needs a Malay title page as well,
%% so you'll need to specify the Malay title, faculty and degreetype.
\title{\mytitle}
\titlems{\mytitlems}
\author{\myauthor}  %% Assuming your name is spelt the same way in English and Malay
\authorid{\myauthorid}
\faculty{\myfaculty}
\facultyms{\myfacultyms}
\submissiondate{\mysubmissiondate}
\submissionyear{\mysubmissionyear}
\degreetype{\mydegreetype}
\degreetypems{\mydegreetypems}
\campus{Bangi} %% Assuming Malaysian cities are spelt the same way in English and Malay

%% If you find the boxes around hyperlinks distracting
\hypersetup{colorlinks,allcolors=black}

\begin{document}

%% Generate the cover page
\makecoverpage

%% Generate the English and Malay title pages.
%% Re-specify your title with different manual line breaks for the English
%% title page, if necessary
\title{\mytitle}
\maketitlepage

%%%% If you're happy to just use the same line-breaking scheme for the English
%%%% title on both the cover and the titlepage, then you can just call
%%%% \maketitle which combines both \makecoverpage and \maketitlepage.

\frontmatter
\declaration

% Acknolwedgements from ack.tex
\input{acknowledgement}

% English abstract from abstract-en.tex
\input{../../abstracts/abstract-en}

% Malay Abstract from abstrak-ms.tex
\input{../../abstracts/abstrak-ms}


\tableofcontents
\listoffigures
\listoftables

% List of Symbols may be prepared as in symbols.tex
\input{../../supplements/symbols}

\mainmatter
% Each chapter from a separate file
\input{chapter1}
\input{chapter2}
\input{chapter3}



% references are listed in references.bib
\bibliography{../../supplements/references}

\appendix
% Each appendix chapter from a separate file
\input{../../supplements/app-details}
\input{../../supplements/app-code}
\end{document}
